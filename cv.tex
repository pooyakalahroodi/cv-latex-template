
\documentclass[a4paper]{mctemplate} % a4paper for A4
\usepackage{graphicx}
\usepackage{calc}


\begin{document}

%----------------------------------------------------------------------------------------
% BACKGROUND SHAPES
%----------------------------------------------------------------------------------------

% Specify sidebar width and header height
\makebackground{7cm}{3.5cm}

%----------------------------------------------------------------------------------------
% HEADER
%----------------------------------------------------------------------------------------

% Specify name, surname and title
\makeheader{Pooya}{Kalahroodi}{}

%----------------------------------------------------------------------------------------
% SIDEBAR
%----------------------------------------------------------------------------------------

\begin{sidebar}

% Specify radius of profile picture
\makepicture{2.1cm}

%-------------------------------%

\begin{contacttable}
    \contactitem{home}{}{Hildesheim, Germany}
	\contactitem{envelope}{mailto:pooya.kalahroodi@gmail.com}{pooyakalahroodi@gmail.com}

	\contactitem{phone}{}{+49 1789771241}
	\contactitem{envelope}{mailto:pooya.kalahroodi@gmail.com}{pooyakalahroodi@gmail.com}

	\contactitem{github}{https://github.com/pooyakalahroodi}{pooyakalahroodi}
	\contactitem{linkedin}{https://www.linkedin.com/in/pooyakalahroodi}{pooyakalahroodi}
\end{contacttable}

%-------------------------------%

\profilesection{Skills}
\begin{skilltable}
	\skillitem
	{\textbf{Machine Learning}: Designing and implementing recurrent, graph, and convolutional neural networks. Advanced probabilistic machine learning models such as Gaussian processes and factorization            models for complex data                analysis and modeling}
	\skillitem
	{\textbf{Statistics and Mathematics}: Computer Engineering Statistics and 
        Probability for data analysis. Modern optimization techniques tailored for machine learning applications to help fine-tune the loss functions in learning algorithms }
	\skillitem
	{\textbf{Software Engineering}: Comprehensive understanding of software    
        engineering principles, database design, and management.
        Familiarity with Software testing methodologies.}
	\skillitem
	{\textbf{Computing}: Parallelization, multi-threading}
\end{skilltable}

%-------------------------------%

\profilesection{Toolbox}
\toolbox{
    {Apache Hadoop $\textbullet$ Apache Spark $\textbullet$  /0.55},
    {SQL $\textbullet$ Git $\textbullet$  $\textbullet$  /0.8},
    {MS Excel $\textbullet$ Access $\textbullet$ Tibco spotfire /1}}

%-------------------------------%

\profilesection{Coding}
\begin{codingtable}
	\codingitem{python.png}
	{\textbf{Python}: numpy, scipy, pandas, sklearn, pytorch,matplotlib}
	
	\codingitem{tools.png}
	{\textbf{Misc}: C++, Visual Basic, Java}
\end{codingtable}


\end{sidebar}



%----------------------------------------------------------------------------------------
% MAIN
%----------------------------------------------------------------------------------------

\begin{main}

\section{Education}

\begin{experiencelist} 
	\experienceitem
    	{2019 - now }
        {M.Sc. in Data Analytics}
        {\href{https://www.uni-hildesheim.de/}{\textbf{University of Hildesheim}}}
        {Institute of Computer Science, Information Systems and Machine Learning Lab (ISMLL) \\
         }
	\experienceitem
    	{2014 – 2018}
        {B.Sc. in Software Engineering}
        {\href{http://araku.ac.ir/en/1}{\textbf{Arak University}, Iran}}
        {Faculty of Engineering - Computer Engineering Campus}
    
\end{experiencelist}

%-------------------------------%

\section{Work Experience}

\begin{experiencelist}
    \experienceitem
        {2023 - now}
        {Master's Thesis Student}
        {\textbf{Tennet TSO}, Bayreuth}
        {(As a member of the Digital Process Excellence Team)
        \begin{itemize}
              \item Implementing Graph Neural Network (GNN) models for air pollution forecasting.
          \end{itemize}
          
          \begin{itemize}
              \item  Fine-tuning the models to improve predictive accuracy.
          \end{itemize}    
         \begin{itemize}
              \item Creating data visualization and reports using Tibco Spotfire to present findings.
        \end{itemize}
          
        }\\
	\experienceitem
    	{2021 - 2023}
        {Working Student}
        {\textbf{Tennet TSO}, Lehrte}
        {(As a member of the Project Engineering and Management and Business Administration Teams)
          \begin{itemize}
              \item Implementation of the KPI that was introduced to calculate the availability of the high-voltage cables.
          \end{itemize}
          
          \begin{itemize}
              \item Supporting the data migration and transformation among different platforms like SAP.
          \end{itemize}         
          
          \begin{itemize}
              \item Historization of the huge Excel worksheet data with the help of MS Access and SQL Server.
          \end{itemize}}\\

          
    \experienceitem
        {2017 - 2018}
        {Freelance Web Designer and Developer}
        
        {Working independently as a UI designer and frontend programmer for small businesses 
        websites 
        \begin{itemize}
        \item Creating responsive website layouts and designs.
        \item using Bootstrap components to enhance website functionality.
        \end{itemize}
}\end{experiencelist}

%-------------------------------%

\section{Research}
\vspace{-.2cm}

\begin{itemize}
    \item \textbf{Customer Classification Using Transaction Behavioral Data}
    \hfill
    {$
    \begin{array}{c}
    \begin{tikzpicture}
        \node[scale=1, maincolor] at (.5,0){\href{https://drive.google.com/file/d/1BUAqRmDYdR8LqjFazfKBWS_4MRtqwgkL/view?usp=drive_link}{\faIcon[regular]{file-pdf}}};
    \end{tikzpicture}
    \end{array}
    $}
    \newline
    This student research project, conducted as part of my academic studies and sponsored by Sparkasse Finanzportal (SFP), focuses on categorizing customers based on their historical transactional behaviors. It represents an approach to analyzing transactional time series data from a classification perspective.
    
    \item \textbf{Analyzing a Chain Store sale records (B.Sc. Project)}
    \hfill
    {$
    \begin{array}{c}
    
    \end{array}
    $}
    \newline
    The dataset includes 4,000 records that contain information about branches, products, and 
    sales. The goal of this project is to find the relationship between the product features and the
    sales volume in each branch of the store. The problem is a good example of using the linear 
    regression method

    \vspace{.3cm}
\end{itemize}

        
%-------------------------------%     
        
\section{Other}
\vspace{-.2cm}
\begin{itemize}
    \item \textbf{Languages}: English (fluent), German (B1), Persian (Native)
    \item \textbf{Willing to relocate}
    \item \textbf{Hobbies}, Playing soccer and swimming, Reading psychology and philosophy books, Learning languages (German), Meeting new people and making friends


\end{itemize}

\end{main}

\end{document} 
